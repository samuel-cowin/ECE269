\documentclass[12pt]{article}
\usepackage{amsmath, bm, amssymb}
\usepackage{tikz, pgfplots}
\pgfplotsset{compat=1.17}
\usepackage[a4paper,bindingoffset=0.2in,left=1in,right=1in,top=1in,bottom=1in,footskip=.25in]{geometry}

\begin{document}
\setlength{\abovedisplayskip}{0pt}
\setlength{\belowdisplayskip}{0pt}
\setlength{\abovedisplayshortskip}{0pt}
\setlength{\belowdisplayshortskip}{0pt}

\title{Notes for ECE269 - Linear Algebra \\
\large Chapter 4}
\author{Sam Cowin}
\maketitle

\section{Linear Equations in Linear Algebra}
\section{Matrix Algebra}
\section{Determinants}
\section{Vector Spaces}
The point of this chapter is to tie in theory from the initial chapters in order to demonstrate the utlity in input/output systems being understood as vector spaces. %
Through geometry and linear algebra, many concepts can be understood from the foundations of the first few chapters. 
\subsection{Vector Spaces and Subspaces}
A vector space is a nonempty set \textit{V} of objects, called vectors, on which are defined two operations, called addition and multiplication by scalars (real numbers), %
subject to the ten axioms on page 192 of the textbook, which must hold for vectors \textbf{u, v, and w} in \textit{V} and for all scalars \textit{c and d}. The paralellogram %
rule from the earlier chapters is useful here to prove the axioms for generic vectors within any space. Doubly infinite sequences of numbers are equivalnt to discrete time %
measured signals that are common within the field of engineering and especially for signal processing. In the same way that discrete time signals can be understood as %
vector spaces, so can polynomials of degree n as well as functions and their compositions. 
\newline
\newline
A subspace of V, H, has the following three properties from the axiom list mentioned previously:
\begin{itemize}
    \item The zero vector of V is in H
    \item H is closed under vector adition. That is, for each \textbf{u and v}, in H, the sum \textbf{u+v} is in H. 
    \item H is closed under multiplication by scalars. That is, for each \textbf{u} in H and each scalar c, the vector multiple \textbf{u}c is in H. 
\end{itemize} 
The zero subspace is a unique identification denoted {\textbf{0}}. 2D is not a subspace of 3D as it is not even a subset. There needs to be equal dimensionality. A vector %
acting like 2D that is in fact a subspace of 3D, is the 2D vector with a third dimension that is 0. Sets need to be through the origin in order to be a subspace. Every %
subspace for the 3D set is either a plane through the origin or a line through the origin except for the set itself. Is a group of vectors is in the vector space V, then %
the span of those vectors is also in the subspace of V. Rewriting a set as the vector equation can demonstrate the dimensionality for which that set is a subspace. %
\subsection{Null Spaces, Column Spaces, and Linear Transformations}
The null space of a matrix is the set of all solutions for the homogeneous equation. Another way to visualize this is the set of all solutions that are mapped into the %
zero vector through the linear transformation  matrix A. The null space of a matrix for a certain dimension set is the subspace of that set. For the null space to be %
a subspace, it must include the zero vector, which is why it is important that the equations are homogeneous. For review, the spanning set of the null space for a matrix %
is found by row reducing the matrix to solve for the free variables. The not free variables are explicitly described with these other variables, and a vector equation is %
written with the free variables as the vectors and the dimensionality described by the number of columns of the original matrix. This is implicitly linearly independent. %
A column space is a subspace for the dimensionality of the rows of the matrix for which the column space is defined. Unless the matrix under question is square, the null %
space and the column space are recognized for different dimensions. Any column of the matrix is in the column space, and to find a vector for the null space all you need to %
do is row reduce and solve, substituting for the free variables. A table outlining the differences between Nul and Col is on page 206 in the text to be modified later. %
\newline
\newline
The kernal of a linear transformation T is the set of all \textbf{u} in V (vector space) such that T\textbf{(u)=0}. The range of T is the set of all vectors in W 
(vector space) of the form T(\textbf{x}) for some x in V. If this T is a matrix transformation, then the null space and column space from before apply. Derivatives are linear
transformations as they can be proven to hold the attributes of linear transformations from chapter 2. Differential equations are just the kernal or the range for the 
particular equation. 
\subsection{Linearly Independent Sets; Bases}
To restate, linear indepedence is when the linear combination of vectors with scalars is the zero vector only with the trivial 
solution of every scalar being zero. Any set containing the zero vector is linearly dependent as well as any set of multiple 
vectors where the vectors are linear combinations of multiples of each other. The basis in V for H has the following properties:
\begin{itemize}
    \item $\beta$ is a linearly independent set
    \item the subspace spanned by $\beta$ coincides with H or H is a span of the components of $\beta$ 
\end{itemize}
The columns of invertible matrices have to be linearly independent and thus for the basis for the dimensionality they describe. The identity 
matrix is a specific form of this in that it is called the standard basis for the dimensionality it describes. Increasing power polynomial 
sets are only equal to zero when the zero polynomial is introduced and thus are a basis for that dimensionality. If a set describes the 
vectors for which the span describes H, H is still described by this set if there is a linear combination of vectors that describe another 
vector in the set - and this vector is then removed. As long as H is not the zero vector, then this new set can form a basis for H. 
Discarding the free variables from a reduced echelon form matrix yields the basis for the matrix. In other words, the pivot columns of 
a matrix form the basis for the column space. The pivot columns of the reduced matrix do not form the basis, however, of the original
matrix. The pivot columns will need to be applied to the original matrix in order to find its basis. 
\subsection{Coordinate Systems}
The purpose of the basis in to implement a coordinate system on a vector space or to product a new view of the vector space if it has a 
defined coordinate system. The coordinates of a vector \textbf{x} relative to the basis $\beta$ (or the $\beta$ coordinates of \textbf{x}) are 
the scalar weights making the linear combination of the $\beta$ values equal to the vector \textbf{x}. These can also be interpretted as the 
standard coordinates relative to the standard basis. The change of coordinates equation is how to move from $\beta$ coordinates back into 
the standard coordinates. This transformation is invertible and thus is one-to-one linear transformation from 2D onto 2D for the text's 
case. Coordinate mapping is useful for taking a vector space and mapping it into a known dimensionality and the isomorphism between the 
two translates into equality of operations in the two spaces. A simple mapping example would be taking polynomials of a certain degree and 
mapping them into higher dimensionality by involving the constants in the matrices. 
\subsection{The Dimension of a Vector Space}
For a given vector space V with basis $\beta$, any set in V containing more than n vectors (dimensionality of $\beta$) will be linearly 
dependent. Every basis of V must have the same number of vectors. If V is spanned by a finite set, then it is said to be finite-dimensional 
containing the number of dimensions as described by the number of vectors in the basis. The dimension for the zero vector space is zero. 
If V is not spanned by a finite set, it is infinite-dimensional. To check the dimensionality of a subspace, check the first column is not 
zero, there are no multiples of each other in the other columns, and that there are no linear combinations that form downstream columns. 
\newline
\newline
If H is a subspace of a finite-dimensional vector space V, any linearly independent set in H can be expanded to the basis and H is finite-
dimensional. Any linearly indepedent set of p elements, corresponding to the dimensionality of the vector space greater than 1, is a basis 
for that space. Alternatively, any set of p elements spanning the vector space is also a basis. The dimension of Nul A is the number of free 
variables in the matrix equation and the dimension of Col A is the number of pivot columns. 
\subsection{Rank}
If two matrices A and B are row equivalent, then their row spaces are the same. If B is in echelon form, the nonzero rows of B form a basis
 for the row space of A as well as for B. The basis for Row A are found from the echelon form of A and taking the nonzero rows. The basis 
for Col A is from finding the pivots and then utilizing the original matrix A. The basis for the Nul A is found by taking the reduced 
echelon form and solving the homogeneous equations for the free variables and using the vector equation with the free variables as the 
vectors. Only Col A utilizes the original matrix in describing its domain. Row operations change row interdependence and thus the reduced
 matrix does not have the same row dependence. 
 \newline
 \newline
 The rank of A is the dimension of the column space of A. The rank of A plus the dimension of the Nul of A is equal to the number of columns 
 in the matrix of A. 
    
\end{document}