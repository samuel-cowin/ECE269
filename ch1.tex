\documentclass[12pt]{article}
\usepackage{amsmath}
\usepackage[a4paper,bindingoffset=0.2in,left=1in,right=1in,top=1in,bottom=1in,footskip=.25in]{geometry}

\begin{document}

\title{Notes for ECE269 - Linear Algebra \\
\large Chapter 1}
\author{Sam Cowin}

\maketitle

\section{Linear Equations in Linear Algebra}
This first chapter will go over the basics of linear equations and foundations of formulating systems of linear equations %
into networks of vectors and matrices for more substanital analysis later in the text.

\subsection{Systems of Linear Equations}
A linear equation is described as follows:
\begin{equation}
    a_1x_1 + a_2x_2 + \dots + a_nx_n = b
\end{equation}
A system of linear equations is one or more linear equations as decribed above involving the same variables. %
Two linear systems are equivalent if the solution set for the two systems is identical. Linear systems are either %
consistent (have one or infinitely many solutions) or inconsistent (no solution).
\newline
\newline
\noindent A matrix is shown below:
$$
\begin{bmatrix}
    2 & 4 & 6 & 0 \\
    1 & 3 & 5 & 1 \\
    7 & 8 & 9 & 2
\end{bmatrix}
$$
\newline
This is an augmented matrix as the values the equations solve to are included as the right most column. The linear equations are represented%
 by the other columns in the matrix, starting with the second to right most column being constant coefficients. From there, the degree of the variables%
 increases by one per column. An \textbf{m x n} matrix indicates m rows and n columns. 
\newline
\newline
To solve a system of linear equations, there are three methods in simplifying system:
\begin{itemize}
    \item Replacing an equation with the sum of itself and the multiple of another equation
    \item Interchanging two equations
    \item Multiplying an equation by a nonzero constant
\end{itemize}
Two matrices are said to be row equivalent if these operations can be used to equate one matrix to another. This translates into the two row equivalent matrices %
having the same solution set. If in reduced form, there is a constradiction in the solution set, then the system of equations is inconsistent (no solution). 

\section{Supplementary Exercises}

\end{document}