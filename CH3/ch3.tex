\documentclass[12pt]{article}
\usepackage{amsmath, bm, amssymb}
\usepackage{tikz, pgfplots}
\pgfplotsset{compat=1.17}
\usepackage[a4paper,bindingoffset=0.2in,left=1in,right=1in,top=1in,bottom=1in,footskip=.25in]{geometry}

\begin{document}
\setlength{\abovedisplayskip}{0pt}
\setlength{\belowdisplayskip}{0pt}
\setlength{\abovedisplayshortskip}{0pt}
\setlength{\belowdisplayshortskip}{0pt}

\title{Notes for ECE269 - Linear Algebra \\
\large Chapter 3}
\author{Sam Cowin}
\maketitle

\section{Determinants}
Although many of the applications determinants were once used for are no longer applicable, they play a key role in many applications still. This chapter outlines %
relevant applications and derives extensions and ideas about determinants. 
\subsection{Introduction to Determinants}
The determinant stems from the row reduction algorithm applied to a \textit{n x n} matrix. This algorithm applied to an "invertible" matrix requires there %
being pivot positions in each column. Due to this, as row reduction is applied, the determinant forms from the linear combination or row operations in the bottom right %
of the matrix. 
\newline
\newline
For simplicity, the determinant of a 3D matrix is shown here before the general case: 
\newline
$$
\Delta=a_{11}\cdot detA_{11}-a_{12}\cdot detA_{12}+a_{13}\cdot detA_{13}
$$
\newline
In this case, the subscripts of the matrices refer to the row/column pair to be deleted when taken into the equation. This can be extended generally for n>2 %
by alternating the determinant being added to the equation and utilizing N-1 determinants where N is the dimensionality of the matrix you are seeking the determinant. %
\newline
$$
detA=a_{11}\cdot detA_{11}-a_{12}\cdot detA_{12}+\dots+(-1)^{1+n}a_{1n}\cdot detA_{1n}
$$
\newline
This method of the determinant is also the cofactor expansion across the first row of the matrix. The cofactor expansion across any row and any column can be used to %
determine the determinant of the matrix. If the matrix is triangular, then the determinant is the product of the main entries down the diagonol of the matrix. 
\subsection{Properties of Determinants}


\end{document}