\documentclass[12pt]{article}
\usepackage{amsmath, bm, amssymb}
\usepackage{tikz, pgfplots}
\pgfplotsset{compat=1.17}
\usepackage[a4paper,bindingoffset=0.2in,left=1in,right=1in,top=1in,bottom=1in,footskip=.25in]{geometry}

\begin{document}
\setlength{\abovedisplayskip}{0pt}
\setlength{\belowdisplayskip}{0pt}
\setlength{\abovedisplayshortskip}{0pt}
\setlength{\belowdisplayshortskip}{0pt}

\title{Notes for ECE269 - Linear Algebra \\
\large Chapter 3}
\author{Sam Cowin}
\maketitle

\section{Linear Equations in Linear Algebra}
\section{Matrix Algebra}
\section{Determinants}
Although many of the applications determinants were once used for are no longer applicable, they play a key role in many applications still. This chapter outlines %
relevant applications and derives extensions and ideas about determinants. 
\subsection{Introduction to Determinants}
The determinant stems from the row reduction algorithm applied to a \textit{n x n} matrix. This algorithm applied to an "invertible" matrix requires there %
being pivot positions in each column. Due to this, as row reduction is applied, the determinant forms from the linear combination or row operations in the bottom right %
of the matrix. 
\newline
\newline
For simplicity, the determinant of a 3D matrix is shown here before the general case: 
\newline
$$
\Delta=a_{11}\cdot detA_{11}-a_{12}\cdot detA_{12}+a_{13}\cdot detA_{13}
$$
\newline
In this case, the subscripts of the matrices refer to the row/column pair to be deleted when taken into the equation. This can be extended generally for n>2 %
by alternating the determinant being added to the equation and utilizing N-1 determinants where N is the dimensionality of the matrix you are seeking the determinant. %
\newline
$$
detA=a_{11}\cdot detA_{11}-a_{12}\cdot detA_{12}+\dots+(-1)^{1+n}a_{1n}\cdot detA_{1n}
$$
\newline
This method of the determinant is also the cofactor expansion across the first row of the matrix. The cofactor expansion across any row and any column can be used to %
determine the determinant of the matrix. If the matrix is triangular, then the determinant is the product of the main entries down the diagonol of the matrix. 
\subsection{Properties of Determinants}
If row operations are applied to a matrix, the determinant changes in the following ways:
\newline
\newline
\newline
\newline
Assuming there is a square matrix.
\begin{itemize}
    \item If a multiple of one row of A is added to another row to form a matrix B, the determinant does not change. 
    \item If two rows are interchanged to produce a new matrix B, then the determinant is the inverse of the determinant of A.
    \item If one row of A is multiplied by k to produce a new row in B, then the determinant of B is the determinant of A scaled by k.
\end{itemize}
Note - just adding one row to another does not change the determinant either. Factoring out a multiple of a row means that this factor needs to be included %
back into the final result. When the matrix is invertible, the determinant is the product of pivot positions multiplied with the number or row interchanges times -1
for a echelon form matrix. This further proves that a matrix is invertible only if the determinant is zero, since the product of the pivot positions would %
yield zero when the matrix is not invertible. Linear dependence is obvious when two columns or rows are the same or there is a zero column or row. Cofactor %
expansion, row reduction to triangular matrices, and using the above theorems speeds up the operations to find a determinant. For a square matrix, the determinant %
of the transpose is equal to the determinant of the original matrix which yields the result that the row and column terms are interchangeable in the above theorem. %
Determinants also have multiplicative properties in that the determinant of the product of two matrices is the product of the individual determinants, while this %
does not hold true for the addition of the determinants
\subsection{Cramer's Rule, Volume, and Linear Transformations}
Cramer's Rule:
\newline
Let A be an invertible \textit{n x n} matrix. For any \textbf{b} in the solution space, the unique solution for \textbf{x} in the matrix equation has its ith entry given %
by the ith column of A replaced by \textbf{b} divided by the determinant of A. For hand computation, this is only really useful for 2D or 3D applications. The inverse of a %
matrix is simply the inverse of the determinant of the matrix multiplied by the adjugate of the matrix (matrix of cofactors used for the determinant). Multiplying the %
adjugate by the initial matrix yields an expression for the determinant multiplied with the identity matrix, which is another form of the theorem just stated. 
\newline
\newline
Is A is a 2x2 matrix, then the area of the parralelogram determined by the columns of A is the determinant of A. This can also be said for the area of the parralelepiped %
determined by the columns of a 3x3 matrix. The areas and volumes of the objects just described can be multiplied with the determinants of 2D or 3D linear transformations %
in order to find the area or volume of the new object. 

\end{document}